\documentclass{article}
\usepackage{a4}

\newcommand{\ask}{\texttt{ask}}
\newcommand{\prop}{\texttt{prop}}
\newcommand{\Prop}{\texttt{Prop}}
\newcommand{\where}{\texttt{where}}
\newcommand{\by}{\leftarrow}

\begin{document}
\title{\ask}
\author{probably some other people and Conor McBride}
\maketitle


\section{Introduction}

The \ask{} system is a programming language and proof assistant (not) implemented (yet) as an idempotent textfile transducer. You feed your pathetic blethering through \ask{} and it will
build parts of your development which are obviously missing and comment out and upon parts of
your development which are obviously bogus. It must be idempotent in that if you feed it its
own output, there should be nothing new to add or take away.

The \ask{} programming language is a total fragment of Haskell with strictly positive
unindexed inductive datatypes, higher order function types, and higher rank type schemes.
The \ask{} proof language is some form of predicate logic with configurable rules.


\section{Propositions}

The keyword \prop{} introduces the declaration of a new proposition former. $\Prop$ is the
type of propositions.

We define a relation as follows
\[
  \begin{array}{l}
    \prop \; R\; \tau\;\ldots\;\where\\
    \;\;\mathit{intro} \\
    \;\;\vdots\\
  \end{array}
\]
with each $\mathit{intro}_i$ given as
\[
  \begin{array}{l}
    R\;p\;\ldots\;\by\;I\;p'\;\ldots\;\where\\
    \;\;\mathit{condition} \\
    \;\;\vdots\\
  \end{array}
\]
There is no need to name an introduction rule if its conclusion
anti-unifies with the conclusion of all other introduction rules.

Relation symbols may be Constructor symbols or any infix symbol (like
types when TypeOperators is enabled.)
Intro rules must be Constructor symbols or :infix.

For examples,
\[
  \begin{array}[t]{l}
    \prop\;\Prop\;\texttt{\&\&}\;\Prop\;\where \\
    \;\;a\, \texttt{\&\&}\,b\;\where \\
    \;\;\;\;a\\
    \;\;\;\;b
  \end{array}
  \qquad
  \begin{array}[t]{l}
    \prop\;\Prop\;\texttt{||}\;\Prop\;\where \\
    \;\;a\, \texttt{||}\,b \;\by\;\texttt{Left}\;\where \\
    \;\;\;\;a\\
    \;\;a\, \texttt{||}\,b \;\by\;\texttt{Right}\;\where \\
    \;\;\;\;b\\
  \end{array}
\]

Of course, conditions may themselves have hypotheses.
\[\begin{array}{l}
  \prop\;\Prop\;\rightarrow\;\Prop\; \where \\
  \;\; a\,\rightarrow\,b\; \where\\
  \;\;\;\;b\; \where\\
    \;\;\;\;\;\;a
    \end{array}
\]



\end{document}